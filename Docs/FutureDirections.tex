%%%%%%%%%%%%%%%%%%%%%%%%%%%%%%%%%%%%%%%%%%%%%%%%%%%%%%%%%%%%%%%%%%%%%%%%%%%%
% FILE    : FutureDirections.tex
% SUBJECT : Document describing ideas for new VTank features.
% AUTHOR  : (C) Copyright 2009 by Vermont Technical College
%
%%%%%%%%%%%%%%%%%%%%%%%%%%%%%%%%%%%%%%%%%%%%%%%%%%%%%%%%%%%%%%%%%%%%%%%%%%%%

\chapter{Future Directions}
\label{futuredirections}

\VTank\ is an ongoing project that we hope will continue to evolve and grow for many years to come. In this chapter we describe some ideas for features that may be worth adding to \VTank, both simple and complicated.

\section{\Client}

This section contains various ideas for new Client features.

\begin{enumerate}

\item \textbf{Instant Messaging}. A built in instant messaging system would allow players to chat with each other in real time during their games. This would be particularly important in any kind of team oriented game but it might also enrich other game types. Players could use a third party instant messaging system but there are several advantages in having on built into \Client\ instead.
\begin{enumerate}
\item Every player would have an IM account on a common system. There would be no need for players to negotiate a common third party IM system.
\item The IM experience could be customized to \VTank. For example, chat rooms for teams could be automatically created. Players could be allowed to speak ``from beyond the grave'' even when they are dead. \VTank\ administrators could be granted automatic operator rights in the IM system.
\item \VTank\ IM traffic could piggyback on other \VTank\ traffic and thus take advantage of the security and firewall configuration afforded to the \VTank\ system overall.
\end{enumerate}

\end{enumerate}