%%%%%%%%%%%%%%%%%%%%%%%%%%%%%%%%%%%%%%%%%%%%%%%%%%%%%%%%%%%%%%%%%%%%%%%%%%%%
% FILE    : MainServer.tex
% SUBJECT : Document describing design issues in the VTank Server.
% AUTHOR  : (C) Copyright 2009 by Vermont Technical College
%
%%%%%%%%%%%%%%%%%%%%%%%%%%%%%%%%%%%%%%%%%%%%%%%%%%%%%%%%%%%%%%%%%%%%%%%%%%%%

\chapter{\MainServer}
\label{mainserver}

\section{Requirements}

\MainServer\ is a central connection point for any type of \VTank\ client. This includes not only the game client, but clients connecting from websites, from management programs such as the \Janitor, and from administrative programs such as Captain \VTank. \MainServer\ will also act as a proxy between the client and the database, which is to say that only \MainServer\ will have direct access to the database. This section describes the main server's responsibility.

\subsection{Functional Requirements}

\MainServer\ acts as an initial meeting place between a regular game client and a game server. Once \MainServer\ is started, it will receive connections from game servers, which will request a spot on the main server's list of game servers. This list is presented to each client as they connect, giving them the option to choose which game server they wish to play on.

Once a client chooses to join a certain game server, the main server will generate a SHA-1 hash from a random number salted with the client's account name and last login date. The hash is then passed to the game server and to the client. This acts as the client's key to start playing on the game server. Even after a client has started playing a game, the main server must continuously monitor it's behavior. The client is responsible for pushing ``keep alive'' messages to the main server in order to prevent it from being kicked due to inactivity.

In addition to game clients, \MainServer\ is responsible for allowing privileged members log in to perform special actions. The following types of clients are allowed to log in:
\begin{enumerate}
\item Users of developer privilege or higher are allowed to view, add, modify or delete in-game equipment (i.e. weapons, armor, projectiles) via \Janitor.
\item Administrative users will log in through Captain \VTank. Administrators are granted the following abilities:
	\begin{enumerate}
	\item View a complete list of clients.
	\item Kick any client for any reason.
	\item Ban any client for any reason.
	\item View, add, modify or delete accounts. (Administrators may not view passwords.)
	\item Disable or enable a game server.
	\item View in-depth statistics or health reports generated by \MainServer.
	\item Any action that \Janitor\ can do.
	\end{enumerate}
\item Public users through non-specific clients may request certain non-privileged information from \MainServer\ (i.e. a list of in-game clients, the current \VTank\ game client version, etc.).
\end{enumerate}

\subsubsection*{Statistics Tracking}

\MainServer\ tracks certain information for each user. There are two categories of information that \MainServer\ tracks:
\begin{enumerate}
	\item Usage-related information, such as login frequency, last login time, creation date, etc..
	\item Game-related information, such as kills, deaths, wins, losses, and ``experience points''.
\end{enumerate}
Usage-related information is for high-level accounting purposes and has no affect on the game. In contrast, game-related information affects the game in a significant way. It is \GameServer's job to calculate how many points are earned for each accomplishment. However, only statistics sent from an ``approved'' \GameServer\ instance are recorded in \MainServer's database.

Each user has a column of statistics called 'points'. Points are used as a measurement unit to determine what rank the user is placed at. Each user's tank is given an individual rank, which is raised by performing well in games to gain points. After certain thresholds as determined by the server, the tank will be promoted to a new rank. Promotion holds the following effect on users:
\begin{enumerate}
	\item New items or tank chassis may be opened up for use.
	\item Certain bonuses may apply during games.
\end{enumerate}
In addition to individual tank ranks, special ranks are also tied to a user's account to measure overall performance (that is, statistics across multiple tanks). Promotions are harder to attain for accounts and their effect on the game is currently undetermined.

Points and promotion thresholds are currently arbitrarily calculated. Approved servers are allowed to set the point values of each type of accomplishment. For instance, \GameServer\ may decide that kills are worth 20 points, flag captures are worth 100 points, and assists are worth 10 points. However, the final calculated point gain for each player is only a ``recommendation'' to \MainServer: \MainServer\ may decide to ignore those values. When \GameServer\ submits it's calculated values, \MainServer\ checks the values for validity and records them into the database. Because \GameServer\ also sends player statistics such as kills and assists, \MainServer\ may decide to completely ignore the point values received by \GameServer\ and calculate things it's own way, and indeed, if \GameServer\ is calculating too little or too many points, \MainServer\ will ignore \GameServer's recommended point calculation.

\subsection{Non-Functional Requirements}

\subsubsection*{Platform}

\MainServer\ is written in Stackless Python and does not depend on certain operating system functions. \MainServer\ will run on any platform that meets the following requirements:
\begin{enumerate}
\item Supports Stackless Python \StacklessPythonVersion.
\item Supports Ice \IceVersion\ (Python \StacklessPythonVersion\ support required).
\item Supports mysql-python, a Python package which supports MySQL communication.
\end{enumerate}

\subsubsection*{Performance}

Because slow performance in the main server does not affect \VTank's gameplay, \MainServer\ does not have strict performance requirements. However, it is preferable that it has less-than-a-second response time. The performance may be slowed considerably if the backend database's performance is abysmal or if it takes too long to communicate with the main server (which can happen if the database is not on the same machine).

The main server requires very little memory and CPU time.

\subsubsection*{Security}

The main server acts as the initial connection point for every client, including game servers, and most actions require authentication from each of these clients, including administrative clients. Therefore, security in \MainServer\ is a high priority.

Every client must use IceSSL (Ice implementation of the secure socket layer) to authenticate with \MainServer. Once authenticated, actions from the client are generally considered to be non-sensitive, so SSL is not necessary once a session has been established. It should, however, be an option for some clients, especially Captain \VTank.

Passwords will be stored as a hash in a MySQL database column. Once \MainServer\ receives the password from the client via IceSSL, it performs a SHA-1 hash on it. Since the hash is considered irreversible, it would be impossible to log in using that hash if one were somehow able to steal it. The database should be on an isolated network with no access allowed except for the main server, so packet sniffing efforts are not likely to produce results.

If a user attempts to log in with one account too many times, the account will be locked temporarily (the exact time will be decided by the main server) and a notification will be sent to administrators. Subsequent attempts on that account or attempts to log in to several different accounts will result in a temporary ban of the client's IP address and administrators will be notified again. The administrator may decide to lock down an account in question pending further investigation.

Users who forget their password may request a new one from the main server. When an account is created, the user supplies an e-mail address for exactly this purpose. A new password is generated and stored in the database (after being hashed), while the password is sent to the user's registered e-mail address. If the user entered an invalid e-mail address, he is forced to contact \VTank\ support.

\subsubsection*{User Characteristics}

\MainServer\ will be run and managed by administrators, who are expected to be competent in the operating system used. The user must be familiar with passing command-line arguments into programs, editing configuration files, and must have some experience securing their system (particularly the firewall configuration).

\subsubsection*{Scale}

The main server's non-strict performance requirements mean that the main server will likely not have to expand into grid computing or server clusters.\footnote{This may change considerably if VTank gains thousands of active users.}

\subsubsection*{Data Formats}

The main server reads a custom configuration file that is formatted like most \filename{*.ini} files: with square-bracketed section names and configuration properties formatted under each section. Configuration properties are formatted like:

\begin{commands}
key = value
\end{commands}

\subsubsection*{Internationalization}

\MainServer\ configuration files, databases, log outputs and debug outputs are all in US English. There are no plans to change this currently.

\section{Design \& Architecture}

Despite the fact that \MainServer\ can handle several different types of clients, \MainServer\ handles all clients inside of one client manager. This is done for two main reasons:
\begin{enumerate}
	\item It's easier to maintain N number of types of clients under one manager.
	\item It's easier to add multiple types of clients.
\end{enumerate}

\newpage
\begin{figure}
	\centering
	\scalebox{0.50}{\includegraphics*{Figures/echelon_architecture.png}}
	\caption{Object architecture of \MainServer.}
	\label{fig:echelon_architecture}
\end{figure}

\section{Compiling}

\MainServer\ is written in Python, which is an interpreted language. Python files (\filename{*.py}) are compiled to \filename{*.pyc} files, but are still interpreted nonetheless. Python is a dynamic language, so the only way to test if the program will successfully run is to provide 100\% coverage via unit testing.\footnote{Please see the section on Testing.}

Eclipse's PyDev plugin is used to develop the Python code. Projects can be imported to Eclipse and executed from there. The program's startup file, ``\MainServer.py'', can be executed either from Eclipse or from the command line. Assuming the Python interpreter is on the PATH, the main server can be executed using the following command from the directory where ``\MainServer.py'' is located:

\begin{commands}
python \MainServer.py -s\footnote{The '-s' parameter is recommended: this dynamically compiles required Slice code.}
\end{commands}

\subsection{Ice}

Ice binaries are available for download at \url{http://zeroc.com}.

\subsubsection*{Windows}

Download the latest installer package for Visual Studio 2008. Install the package and make a note of where it's installed. Set the environment variable ICEROOT to the location where Ice was installed (information on how to do that is not covered in this document). Some compile scripts use the ICEROOT variable to locate certain binary or required slice files.

Be sure to add the \filename{\%ICEROOT\%/python/} directory to the PYTHONPATH variable. These files are needed by Python to use Ice modules. Test the Ice implementation of Python by opening the Python interpreter and running these commands:

\begin{lstlisting}
import Ice
Ice.initialize()
\end{lstlisting}

An exception will be thrown from one of these statements if Ice is not working correctly.

\subsection{Stackless Python}

Stackless is a third-party implementation of Python. It installs by itself and works exactly the same way as the regular Python installation, but adds a module named ``stackless''. A more detailed explanation on Stackless and a download is available at it's website: \url{http://stackless.com}.

\subsection{mysql-python}

mysql-python is a third-party module which assists in executing SQL statements on a MySQL database. It's available for download at: \url{https://sourceforge.net/projects/mysql-python}.

\subsubsection*{Windows}

It's very simple to install mysql-python on Windows. Simply download the latest package and execute the installer. The installer will find the latest compatible version of Python and install itself there. To test if it works, open the Python interpreter and type:

\begin{lstlisting}
import MySQLdb;
\end{lstlisting}

If an ImportError does not occur, then the package installed successfully.

\subsubsection*{Linux}

Use the following simple procedure to install mysql-python on a Linux machine:

\begin{enumerate}
\item Download the latest source of MySQL-python from \url{http://sourceforge.net/projects/mysql-python}. The source is stored as a \filename{.tar.gz} file.
\item Do ``tar xfz MySQL-python-1.2.3.tar.gz'' (being sure to use the correct version number)
\item Do ``cd MySQL-python-1.2.3'' (being sure to use the correct version number)
\item Edit site.cfg if necessary.
\item Execute: ``sudo python setup.py build''
\item Execute: ``sudo python setup.py install''
\end{enumerate}

To test if it works, open the Python interpreter and type:

\begin{lstlisting}
import MySQLdb;
\end{lstlisting}

If an ImportError does not occur, then the package installed successfully.